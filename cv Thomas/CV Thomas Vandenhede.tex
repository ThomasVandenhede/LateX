\documentclass[10pt, a4paper]{article}
\usepackage{marvosym}
\usepackage{fontspec} % fonts
\usepackage{fontawesome} % web icons (load AFTER fontspec)
\usepackage{setspace}
\usepackage{xunicode, xltxtra, url, parskip}
\defaultfontfeatures{Scale=MatchLowercase, Mapping=tex-text}
\RequirePackage{color, graphics}
\usepackage[usenames, dvipsnames]{xcolor}
\usepackage[left=2.5cm, right=2.5cm, top=1.2cm, bottom=1.5cm]{geometry}
\usepackage{supertabular}
\usepackage{titlesec}
\usepackage{multicol}
\usepackage{multirow}
\usepackage{longtable}
\usepackage{tabu}
\usepackage{xstring}
\usepackage{ifthen}
\usepackage{enumitem}
\usepackage[xetex, unicode, pdfencoding=auto]{hyperref}
\usepackage[absolute]{textpos}
\usepackage{enumitem}
\usepackage{tabularx}
\usepackage{soul}% http://ctan.org/pkg/soul -> underlining features

% \makeatletter
% \renewcommand*{\@biblabel}[1]{\hfill[#1]}
% \makeatother

% Redefine fontawesome social icons
\newfontfamily{\FA}{[FontAwesome.otf]}
\def\TwitterIcon{{\FA \faTwitter}}
\def\GithubIcon{{\FA \faGithubAlt}}
\def\CodepenIcon{{\FA \faCodepen}}
\def\EnvelopeIcon{{\FA \faEnvelopeO}}
\def\MapMarkerIcon{{\FA \faMapMarker}}
\def\PhoneIcon{{\FA \faPhone}}
\def\LinkedinIcon{{\FA \faLinkedin}}
\def\CodeIcon{{\FA \faCode}}

% Configure default underline spacing
\setul{1pt}{.4pt}% 1pt below contents

% Template obtained from http://www.cv-templates.info/2009/03/professional-cv-latex/

%Setup hyperref package, and colours for links
\definecolor{linkcolour}{rgb}{0,0.2,0.6}
\hypersetup{colorlinks,breaklinks,urlcolor=linkcolour, linkcolor=linkcolour}

%Colors
\definecolor{lightg}{HTML}{999999}
\definecolor{medg}{HTML}{666666}
\definecolor{darkg}{HTML}{333333}
\definecolor{noteone}{HTML}{999999}
\definecolor{notetwo}{HTML}{848484}
\definecolor{notethree}{HTML}{424242}
\definecolor{notefour}{HTML}{212121}
\definecolor{notefive}{HTML}{000000}

% Bullets
\newcommand{\fivenotes}{
	\textcolor{noteone}{\symbol{"2022}}
	\textcolor{notetwo}{\symbol{"2022}}
	\textcolor{notethree}{\symbol{"2022}}
	\textcolor{notefour}{\symbol{"2022}}
	\textcolor{notefive}{\symbol{"2022}}
}
\newcommand{\fournotes}{
	\textcolor{noteone}{\symbol{"2022}}
	\textcolor{notetwo}{\symbol{"2022}}
	\textcolor{notethree}{\symbol{"2022}}
	\textcolor{notefour}{\symbol{"2022}}
	\textcolor{white}{\symbol{"2022}}
}
\newcommand{\threenotes}{
	\textcolor{noteone}{\symbol{"2022}}
	\textcolor{notetwo}{\symbol{"2022}}
	\textcolor{notethree}{\symbol{"2022}}
	\textcolor{white}{\symbol{"2022}}
	\textcolor{white}{\symbol{"2022}}
}
\newcommand{\twonotes}{
	\textcolor{noteone}{\symbol{"2022}}
	\textcolor{notetwo}{\symbol{"2022}}
	\textcolor{white}{\symbol{"2022}}
	\textcolor{white}{\symbol{"2022}}
	\textcolor{white}{\symbol{"2022}}
}
\newcommand{\onenote}{
	\textcolor{noteone}{\symbol{"2022}}
	\textcolor{white}{\symbol{"2022}}
	\textcolor{white}{\symbol{"2022}}
	\textcolor{white}{\symbol{"2022}}
	\textcolor{white}{\symbol{"2022}}
}

\newcommand{\oneskill}{
  \textcolor{white}{\symbol{"2022}}
  \textcolor{white}{\symbol{"2022}}
  \textcolor{notefive}{\symbol{"2022}}
}

\newcommand{\twoskill}{
  \textcolor{white}{\symbol{"2022}}
  \textcolor{notethree}{\symbol{"2022}}
  \textcolor{notefive}{\symbol{"2022}}
}

\newcommand{\threeskill}{
  \textcolor{noteone}{\symbol{"2022}}
  \textcolor{notethree}{\symbol{"2022}}
  \textcolor{notefive}{\symbol{"2022}}
}

%FONTS
\defaultfontfeatures{Mapping=tex-text}
\setmainfont{Fontin}[SmallCapsFont = Fontin SmallCaps]

\setromanfont{Linux Libertine}[Ligatures={Common}, BoldFont={Linux Libertine Bold}, ItalicFont={Linux Libertine Italic}]

\setsansfont{GeosansLight}[Ligatures={Common}, BoldFont={GeosansLight}, ItalicFont={GeosansLight}]

%\setmonofont{GeosansLight} --> replaced with following statement
\setmonofont{Consolas}

\font\lighttext=''LibreBaskerville-Regular:color=787878'' at 10pt
\font\lighttextweb=''LibreBaskerville-Regular:color=FF1493'' at 10pt

%CV Sections inspired by:
%http://stefano.italians.nl/archives/26
\titleformat{\section}{\Large\scshape\raggedright}{}{0em}{}[\titlerule]
\titlespacing{\section}{0pt}{4pt}{3pt}
\setlength{\parindent}{0pt}

\newenvironment{education}
{
	\begin{tabu} to \textwidth {@{} X[1,r] | X[5,l] @{}}
}
{
	\end{tabu}
}
\newcommand{\eduentry}[2]{
	\textsc{#1} & \begin{tabu}{@{} X @{}}
		#2
	\end{tabu}\\
	\multicolumn{2}{c}{}\\ [-1ex]
}
\newcommand{\degree}[1]{\textbf{#1}}
\newcommand{\institution}[1]{\textsc{#1}}
\newenvironment{experiences}
{
	\begin{tabu} to \textwidth {@{} X[1,r] | X[5,l] @{}}
}
{
	\end{tabu}
}
\newcommand{\experience}[5]{
	\textsc{#1} & \begin{tabu}{@{} X @{}}
		\textbf{#2} #3 \textsc{#4}\\
		#5
	\end{tabu}\\
	\multicolumn{2}{c}{}\\ [-1ex]
}
\newenvironment{skillslisting}
{
	\begin{multicols}{4}
	\raggedcolumns
	\begin{itemize}
	\renewcommand{\labelitemi}{}
	\renewcommand{\skill}{\textnormal}
	\setlength{\itemsep}{0pt}
	\setlength{\parskip}{0pt}
	\setlength{\parsep}{0pt}
}
{
	\end{itemize}
	\end{multicols}
}

\newcommand{\skills}[2]{
	\item #1 #2
}
\newcommand{\skill}{\textbf}
\newcommand{\proj}[3]{
	\textsc{#1} & #2\\
	&\href{http://www.#3}{#3}\\
	\multicolumn{2}{c}{} \\ [-1ex]
}
\newcommand{\projl}[3]{
	\textsc{#1} & #2\\
	&\href{http://www.#3}{#3}\\
}
\newcommand{\projlh}[4]{
	\textsc{#1} & #2\\
	&\href{#3}{#4}\\
}

% \def\bullet{\textcolor{medg}{\symbol{"00BB}}}
\def\div{\,\textbar{}\,}

% language macros
\usepackage{xspace}

\newcommand{\lang}[2]{\expandafter\def\csname #1\endcsname{\skill{#2}\xspace}}

\lang{ajax}{AJAX / JSON}
\lang{angular}{AngularJS}
\lang{bootstrap}{Bootstrap}
\lang{js}{JavaScript}
\lang{backbone}{Backbone.js}
\lang{nodejs}{Node / Express}
\lang{python}{Python}
\lang{ruby}{Ruby}
\lang{java}{Java}
\lang{matlab}{MATLAB}
\lang{bash}{Bash}
\lang{c}{C}
\lang{clojure}{Clojure}
\lang{css}{CSS 3}
\lang{cascalog}{Cascalog}
\lang{hadoop}{Hadoop}
\lang{spark}{Spark}
\lang{cpp}{C++}
\lang{ccpp}{C / C++}
\lang{scheme}{Scheme}
\lang{django}{Django}
\lang{mongo}{MongoDB}
\lang{numpy}{NumPy}
\lang{scipy}{SciPy}
\lang{opencv}{OpenCV}
\lang{jquery}{jQuery}
\lang{git}{Git / GitHub}
\lang{linux}{Linux}
\lang{android}{Android}
\lang{html}{HTML 5}
\lang{visb}{Visual Basic}
\lang{haskell}{Haskell}
\lang{office}{Microsoft Office}
\lang{php}{PHP}
\lang{react}{ReactJS}
\lang{redux}{Redux}
\lang{refac}{Refactoring}
\lang{patterns}{Design Patterns}
\lang{qt}{Qt}
\lang{sass}{CSS 3 / Sass}
\lang{sql}{SQL}
\lang{salesforce}{Salesforce}
\lang{postgres}{PostgreSQL}
\lang{wcag}{WCAG}
\lang{wordpress}{WordPress}

\pagestyle{empty}

\begin{document}

{\Large \textbf{Thomas Vandenhede}}\smallskip\\
\begin{minipage}{.5\textwidth}
	\begin{tabularx}{\linewidth}{@{}c l@{}}
		\MapMarkerIcon	& 50 avenue Charles Tellier\\
						& 78800 Houilles\\
		\PhoneIcon		& 06 59 27 88 42\\
		\EnvelopeIcon	& \href{mailto: thomas@vandenhede.me}{thomas@vandenhede.me}\\
	\end{tabularx}
\end{minipage}%
\begin{minipage}{.5\textwidth}
	\begin{tabularx}{\linewidth}{@{}X c l@{}}
		& \CodeIcon		& \href{http://vandenhede.me}{vandenhede.me}\\
		& \GithubIcon	& \href{https://github.com/ThomasVandenhede/}{github.com/ThomasVandenhede/}\\
		& \CodepenIcon	& \href{https://codepen.io/ThomasVandenhede/}{codepen.io/ThomasVandenhede/}\\
		& \LinkedinIcon	& \href{https://www.linkedin.com/in/ThomasVandenhede/}{linkedin.com/in/ThomasVandenhede/}\\
	\end{tabularx}
\end{minipage}%
\bigskip
\begin{center}
	\huge \textbf{Développeur Front-End}
\end{center}%
\section{Formation}
\begin{education}
	\eduentry{Mars 2018}{
		\degree{Formation Développement Web Full Stack JS},~\institution{\href{https://www.ifocop.fr/}{Ifocop}},~Paris XI\\
		\hangindent=1.0em --~Rentrée le \textbf{16 mars 2018}. Contrat de professionalisation en alternance.\\
		\hangindent=1.0em --~Programme de la formation et plaquette de l'alternance : \href{https://goo.gl/9aCfZv}{https://goo.gl/9aCfZv}
	}
	\eduentry{Octobre 2016 - Présent}{
		\degree{Autoformation au métier de développeur web}\\
		\hangindent=1.0em --~Cours en ligne : \href{www.codecademy.com}{www.codecademy.com} / \href{www.codeschool.com}{www.codeschool.com}\\
		\hangindent=1.0em --~Lectures personnelles. Liste disponible à l'adresse : \href{http://amzn.eu/g28QJrg}{http://amzn.eu/g28QJrg}
	}
	\eduentry{Septembre 2016}{
		\degree{Master 2 \'Energétique et Matériaux},~\institution{Université Paris Ouest Nanterre La Défense}, Ville d'Avray
	}
	\eduentry{2009 --- 2012}{
		\degree{1re et 2e années d'école d’ingénieurs -- Modélisation Mathématique et Mécanique},~\institution{Enseirb-Matmeca},~Bordeaux
	}
	\eduentry{2006 --- 2009}{
		\degree{CPGE MPSI / MP},~\institution{Lycée Claude Bernard},~Paris
	}
	\eduentry{2006}{
		\degree{Baccalauréat S},~\institution{Lycée Les Pierres Vives},~Carrières-sur-Seine
	}
\end{education}
\vspace{-1em}
\section{Expérience Professionnelle}
\begin{experiences}
	\experience{Février 2016 Août 2016}
	{Stage Modélisation Environnementale (Acoustique \& \'Emissions)}
	{\hfill}
	{\href{http://www.env-isa.com}{Envisa}}
	{\hangindent=1.0em --~Développement de scripts en langage \python pour l'automatisation de la collecte et de l'analyse de données bruit et émissions.

	\hangindent=1.0em --~Logiciel \python de calcul des émissions des engins de piste pour les aéroports dans le cadre du projet européen Open-ALAQS. Utilisation de bases de données \sql (SQLite) et du framework \qt (PyQt).
	
	\hangindent=1.0em --~Administration de site web \wordpress.}
	\experience{Mai 2013 Juillet 2013}
	{Stage Modélisation de Coûts de Maintenance Aéronautique}
	{\hfill}
	{\href{http://www.safran-transmission-systems.com}{Hispano-Suiza}}
	{\hangindent=1.0em --~Développement, au cours d'un stage de 3 mois, d'un outil de modélisation des coûts de maintenance aéronautique sous Excel (\office) faisant appel à \visb (VBA).}
\end{experiences}
\vspace{-1em}
\section{Compétences Informatiques \& Langages}
\vspace{-0.5em}
\begin{skillslisting}
	\skills{\threeskill}{\python}
	\skills{\twoskill}{\bootstrap}
	\skills{\threeskill}{\html}
	\skills{\threeskill}{\sass}
	\skills{\threeskill}{\js}
	\skills{\threeskill}{\jquery}
	\skills{\twoskill}{\react}
	\skills{\oneskill}{\angular}
	\skills{\threeskill}{\sql}
	\skills{\twoskill}{\git}
	\skills{\twoskill}{\refac}
	\skills{\twoskill}{\patterns}
	\skills{\oneskill}{GNU / \linux}
	\skills{\oneskill}{\visb}
	\skills{\twoskill}{\LaTeX}
	\skills{\twoskill}{\matlab}
\end{skillslisting}
\vspace{-1em}
\begin{footnotesize}
	\oneskill Basique \hfill
	\twoskill Intermédiaire \hfill
	\threeskill Avancé
\end{footnotesize}
\section{Langues}
\begin{tabu} to \textwidth {@{} X[1,r] X[5,l] @{}}
	\textbf{Français} & Langue maternelle\\
	\textbf{Anglais} & Lu, parlé et écrit avec aisance \em (Score TOEIC : 985 sur 990)\\
	\textbf{Espagnol} & Niveau scolaire\\
	\textbf{Italien} & Notions\\
	\textbf{Russe} & Notions. \em (Séjour 1 mois à Moscou, certificat d'étude délivré par l'Institut d'état de langue russe A.S. Pouchkine)
\end{tabu}

%\section{Personal and Open Source Projects}
%\begin{tabularx}{\textwidth}{@{}p{3cm}|X@{}}
%	\proj{matsciseg}
%	{Framework for propagated 3D volume segmentation, used in my
%	dissertation work.  Algorithms created in \python and \cpp and
%	exposed as a web API using \django. Includes a web application
%	that consumes the API created in \js, and \jquery.}
%	{github.com/malloc47/matsciseg}
%\end{tabularx}

\section{Activités \& Centres d'Intérêt}
Programmation orientée objet, programmation web, langues étrangères, tennis.

\end{document}
